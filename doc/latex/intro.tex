% !TEX root = sw.tex
% !TeX spellcheck = en_US

\newcommand{\sectionToc}[1]{%
    \section*{#1}%
    \addcontentsline{toc}{section}{#1}%
}


\chapter*{Introduction}
\addcontentsline{toc}{chapter}{Introduction}

\sectionToc{This document}

This document describes a modern approach to software management.
It consists of several chapters which include the new approach itself, documentation for client-side tools and utilities and server-side (website) functionality description.

License for this documentation is unspecified yet.
% GNU FDL?
% CC?

\sectionToc{Overview}

Software Network is a project dedicated to better software management.

Originally started as a package manager (CPPAN - \url{https://github.com/cppan/cppan}) to C/C++ languages based on CMake build system, it is evolved to independent set of tools and libraries.
CPPAN or v1 was a playground where different ideas were studied and checked.

Main user tool is called `sw`. Whole project also may be called as SW. Pronounce it as you like: `[software]` or `[sw]` or `[sv]`.



\sectionToc{History}

\subsection*{CPPAN or SW v1}

Idea was formed during years of work with CMake, different projects, porting things to different OSs.
Development started in the early 2016.

YAML was choosen as configuration file format. It was mixed declarative approach with imperative CMake insertions.

\subsection*{SW or CPPAN v2}

Development started in the late 2017.

Decarative approach showed weaknesses, so it was decided to switch to the most flexible thing - complete programming language. C++ was choosen over Python (Conan) and, possibly, other languages such as Lua (Premake).


\section*{More on CPPAN}

CPPAN -- C++ Archive Network.
This section is not related to current status of SW and given for historical reasons.

The idea comes from:

\begin{enumerate}
\item
Comprehensive Perl Archive Network (CPAN), CRAN (R-language), CTAN (TeX).
\item
Java packages and build systems (Maven).
\item
C++ Modules proposals and presentations by Gabriel Dos Reis.
\end{enumerate}

In the beginning project aimed on C++ project with Modules only. So, the project should evolve by their release. But during development, CPPAN shows great capabilities of handling current C++98/C++11/C++14 projects and even some C libraries.

General principles of CPPAN are listed below.

\begin{enumerate}
\item
Source code only! You do not include your other stuff like tests, benchmarks, utilities etc. Only headers and sources (if any). On exception here: project's license file. Include it if you have one in the project tree.

\item
Semantic versioning http://semver.org.

\item
Zero-configure (zero-conf.) projects. Projects should contain their configurations in headers (relying on toolchain macros) or rely on CPPAN utilities (macros, different checkers in configuration file) or have no config steps at all (header only projects). Still CPPAN provides inlining of user configuration steps, compiler flags etc.

\item
All or Nothing rule for dependencies. Many projects have optional dependencies. In CPPAN they should list them and they'll be always included to the build or not included at all. So, no optional dependencies.
\end{enumerate}

Projects' naming

Project names are like Modules from this C++ proposal http://open-std.org/JTC1/SC22/WG21/docs/papers/2016/p0143r1.pdf. Words are delimeted by points `.`.

Root names are:

1. pvt - for users' projects. E.g. `pvt.cppan.some_lib`.
2. org - for organization projects with open source licenses. E.g. `org.boost.algorithm`.
3. com - for organization projects with commercial licenses. E.g. `com.some_org.some_proprietary_lib`.

Project versions

Each project will have its versions. Version can be a semver number `1.2.8` or a branch name `master`, `dev`. Branches always considered as unstable unless it is stated by maintainer. Branches can be updated to the latest commit. And fixed versions can not.



\sectionToc{Useful links}


\begin{enumerate}
\item
Website - \url{https://software-network.org/}

\item
Download client - \url{https://software-network.org/client/}
\end{enumerate}


\section*{Acknowledgements}

\undef\sectionToc
