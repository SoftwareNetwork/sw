% !TEX root = sw.tex
% !TeX spellcheck = en_US

\newcommand{\sectionToc}[1]{%
    \section*{#1}%
    \addcontentsline{toc}{section}{#1}%
}


\chapter*{Introduction}
\addcontentsline{toc}{chapter}{Introduction}

\sectionToc{This document}

This document describes a modern approach to software management.
It consists of several chapters which include the new approach itself, documentation for client-side tools and utilities and server-side (website) functionality description.

License for this documentation is unspecified yet.
% GNU FDL?
% CC?

\sectionToc{Overview}

Software Network is a project dedicated to better software management.

Originally started as a package manager (CPPAN - \url{https://github.com/cppan/cppan}) to C/C++ languages based on CMake build system, it is evolved to independent set of tools and libraries.
CPPAN or v1 was a playground where different ideas were studied and checked.

Main user tool is called `sw`. Whole project also may be called as SW. Pronounce it as you like: `[software]` or `[sw]` or `[sv]`.



\sectionToc{History}

\subsection*{CPPAN or SW v1}

Idea was formed during years of work with CMake, different projects, porting things to different OSs.
Development started in the early 2016.

YAML was choosen as configuration file format. It was mixed declarative approach with imperative CMake insertions.

\subsection*{SW or CPPAN v2}

Development started in the late 2017.

Decarative approach showed weaknesses, so it was decided to switch to the most flexible thing - complete programming language. C++ was choosen over Python (Conan) and, possibly, other languages such as Lua (Premake).



\sectionToc{Useful links}


\begin{enumerate}
\item
Website - \url{https://software-network.org/}

\item
Download client - \url{https://software-network.org/client/}
\end{enumerate}


\section*{Acknowledgements}

\undef\sectionToc
